\documentclass[12pt, letterpaper]{article}
\usepackage{graphicx}
\usepackage{caption}
\usepackage[hidelinks]{hyperref}
\graphicspath{{images/}}

\begin{document}
    \begin{titlepage}
    \begin{center}
        \vspace*{1cm}
            
        \Huge
        \textbf{Project Modeling Report}
            
        \vspace{0.5cm}
        \LARGE
        Thermal Scanning App
            
        \vspace{1.5cm}
            
        \textbf{Colter Roche, Jose Bastardo}
            
        \vfill
          
        \Large
        Senior Design 1\\
        COP4934C.01\\
        \today
            
    \end{center}
\end{titlepage}
    \newpage
    \tableofcontents
    \newpage
    \listoffigures
    \newpage
    \section{Project Background}
    \paragraph{}
    Corserva is a managed IT service provider that develops and sells custom software and hardware solutions. Corserva's customers
    include hospitality and other in-person focused related businesses. Official CDC guidelines to businesses encourage taking steps to 
    prevent the spread of Covid-19 among employees and customers, including temperature checks. Corserva has sponsored this project 
    to produce a thermal screening solution capable of processing people quickly and without requiring user interaction to minimize additional
    contact.
    \paragraph{}
    Preliminary development and testing has already been completed by Corserva. The system uses a thermal and a regular camera to 
    recognize users and read the temperature of their face. If the temperature is above a certain threshold, then an alert is shown and 
    the flagged person can be moved aside for further screening.  The first iteration of the thermal app has several problems:
    \begin{itemize}
        \item The camera suffers from temperature drift over time and required frequent callibration.
        \item The facial recognition process is around 5 seconds, limiting the rate people can be processed.
        \item Onboarding new users requires interaction directly through the app, not practical for large numbers of employees.
        \item No option to send an alert to the scanned user.
    \end{itemize}
    The new application developed as part of this project will address these problems.
    \section{Scope}
    \subsection{Project Scope}
    \paragraph{}
    The scope of this project is to produce an application and companion mobile application 
    to measure and report high temperatures of people passing through the system. The 
    thermal camera will use an auto calibration system to increase accuracy of 
    readings. Mobile application to smooth the onboarding process and provide reports 
    to users.
    \subsection{Business Scope}
    \paragraph{}
    The business scope is to provide business with a kiosk and mobile app system that will make 
    it easier for them to maintain safety precautions during the current Covid 19 
    pandemic while also increasing the speed in which staff and customers can enter their place of business.
    \section{System Overview}
    \subsection{Users}
    Those who will benefit and be affected by the new solution include:
    \begin{itemize}
        \item Attendant - The new app will no longer require frequent calibration, lessening the need for supervision and technical troubleshooting.
        \item Corserva - The app will be well structured and extensible, allowing for relative ease when adding new features.
        \item Customers - The streamlined onboarding system for users will alleviate concerns to do with overcrowding and long wait times to register in person. 
        \item End Users - Users will receive a notification and report in the companion app after they are scanned, increasing their health awareness.
    \end{itemize}
    \subsection{Location}
    \paragraph{}
    The system can be installed anywhere with electrical power and space for a callibration object. The companion app
    will be available for Android and IOS.
    \subsection{Responsibilities}
    \paragraph{}
    The primary responsibilities of the new system are:
    \begin{itemize}
        \item 
    \end{itemize}
    Other desired features include:
    \begin{itemize}
        \item 
    \end{itemize}
    The system will not be responsible for 
    \section{User Requirements}
    \subsection{Interfaces}
    \subsection{Outputs}
    \subsection{Inputs}
    \subsection{Processing}
    \subsection{Data Storage}
    \section{Functional Decomposition}
    \section{Data Flow and Application Structure}


\end{document}